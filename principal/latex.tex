% para comentar use esse símbolo de porcentagem

\documentclass[a4paper,12pt]{article}

\usepackage[utf8]{inputenc}
\usepackage[T1]{fontenc}
\usepackage{geometry}

\usepackage[francais]{babel}

\title{Como escrever em \LaTeX\ ?}
\author{Ciência da Computação}
\date{}

\begin{document}
\maketitle

Como escrever em \LaTeX\ 

Para escrever em itálico:\\
\textbackslash textit \{texto\} (\textbf{i}talic)
\[\textit {texto}\]

Para escrever em negrito:\\
\textbackslash textbf  \{texto\} (\textbf{b}old\textbf{f}ace)
\[\textbf {texto}\]

Para escrever letras gregas: \\
\$\textbackslash alpha\textbackslash beta\textbackslash gamma\$ \\
\$\textbackslash Sigma\textbackslash Delta\textbackslash Omega\$\\

\[\alpha \beta \gamma\] 
\[\Sigma \Delta \Omega\]

Para escrever uma fórmula: \\
\$fórmula\$
\[f(x)= x^2\]

Para sobreescrever: \\
\$texto\^{}\{texto sobreescrito\}\$
\[x^{2k+1}\]

Para subescrever: \\
\$texto\_\{texto sobreescrito\}\$ 
\[x_1\]

Mutiplicação: \\
\$fator \textbackslash times fator = resultado\$ 
\[3 \times 3 = 9\]

Divisão: \\
\$fator \textbackslash div fator = resultado\$ 
 \[4 \div 2 = 2\]

Frações: \\
\$\textbackslash frac\{numerador\}\{denominador\}\$
\[\frac{1}{2}\]

Raízes: \\
\$\textbackslash sqrt [índice] \{radicando\}\$
\[\sqrt [n] {2}\]
\end{document}

