% para comentar use esse símbolo de porcentagem

\documentclass[a4paper,12pt]{article}

\usepackage[utf8]{inputenc}
\usepackage[T1]{fontenc}
\usepackage{geometry}

\usepackage[francais]{babel}

\title{Como escrever em \LaTeX\ ?}
\author{
	\textsc{Ernesto Yuiti Saito}\\
	\textsc{Hayato Fujii}\\
	\textsc{Hélio Albano de Oliveira}\\
	\textsc{Luiz Henrique Jofre da Silva}\\
	\textsc{Marcos Okamura Rodrigues}\\
}
\date{Londrina 2009}

\begin{document}

\maketitle

% Espaçamento 1,5 entre linhas
\renewcommand{\baselinestretch}{1.5}

\hspace{3cm}Os seguintes comandos têm um significado especial no \LaTeX\ :
\begin{itemize}
\item \textbackslash \ especifica um comando;
\item \& separa colunas;
\item \$ especifica comandos matemáticos;
\item \% ignora linha (comentários);
\item \^{} potência (sobreescrito);
\item \_  subíndice (subscrito);
\item \{\} parte dos comandos;
\end{itemize}

\hspace{3cm}Para inicializar um ambiente:
\begin{verbatim}
\begin{ambiente}
Seu texto
\end{ambiente}
\end{verbatim}

\hspace{3cm}Para criar um documento:
\begin{verbatim}
\begin{document}
Seu documento
\end{document}
\end{verbatim}

\hspace{3cm}Para criar uma lista:
\begin{verbatim}
\begin{tipo de lista}
\item item1;
\item item2;
.............
\item item n;
\end{tipo de lista}
\end{verbatim}

\hspace{3cm}Para escrever em itálico: (\textbf{i}talic)
\begin{verbatim}
\textit{texto} 
\end{verbatim}
\[\textit {texto}\]

\hspace{3cm}Para escrever em negrito: (\textbf{b}old\textbf{f}ace)
\begin{verbatim}
\textbf{texto} 
\end{verbatim}
\[\textbf {texto}\]

\hspace{3cm}Para escrever letras gregas: 
\begin{verbatim}
$\alpha\ beta\ gamma$                   
$ \Sigma\ Delta\ Omega$
\end{verbatim}
\[\alpha \beta \gamma\] 
\[\Sigma \Delta \Omega\]

\hspace{3cm}Para escrever uma fórmula: 
\begin{verbatim}
$fórmula$
\end{verbatim}
\[f(x)= x^2\]

\hspace{3cm}Para sobreescrever: 
\begin{verbatim}
$texto^{}{texto sobreescrito}$
\end{verbatim}
\[x^{2k+1}\]

\hspace{3cm}Para subescrever: 
\begin{verbatim}
$texto_{texto sobreescrito}$ 
\end{verbatim}
\[x_1\]

\hspace{3cm}Mutiplicação: 
\begin{verbatim}
$fator \times fator = resultado$ 
\end{verbatim}
\[3 \times 3 = 9\]

\hspace{3cm}Divisão: 
\begin{verbatim}
\fator \div fator = resultado$ 
\end{verbatim} 
\[4 \div 2 = 2\]

\hspace{3cm}Frações: 
\begin{verbatim}
$\frac{numerador}{denominador}$
\end{verbatim}
\[\frac{1}{2}\]

\hspace{3cm}Raízes: 
\begin{verbatim}
$\sqrt[índice]{radicando}$
\end{verbatim}
\[\sqrt [n] {2}\]

\end{document}

