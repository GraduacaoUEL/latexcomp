% para comentar use esse símbolo de porcentagem

% Preâmbulo
\documentclass[a4paper,12pt]{article}

\usepackage[utf8]{inputenc}
\usepackage[T1]{fontenc}
\usepackage{geometry}

\usepackage[brazil]{babel}
%\usepackage[T1]{fontenc}
\usepackage{ae}

\DeclareOption{brazil}{\input{portuges.ldf}}

\title{Como escrever em \LaTeX\ }
\author{
	Ernesto Yuiti Saito\\
	Hayato Fujii\\
	Helio Albano de Oliveira\\
	Luiz Henrique Jofre da Silva\\
	Marcos Okamura Rodrigues\\
}
\date{\today}

% Fim preâmbulo

% Início documento em si
\begin{document}

% Faça o título conforme definido no title-preâmbulo
\maketitle

% Espaçamento 1,5 entre linhas
%\renewcommand{\baselinestretch}{1.5}

\section{Observações}
Os seguintes caracteres têm um significado especial no \LaTeX\ :
\begin{itemize}
\item \textbackslash \ especifica um comando;
\item \& separa colunas;
\item \$ especifica comandos matemáticos;
\item \% ignora linha (comentários);
\item \^{} escreve em sobreescrito. Útil para escrever potências, por exemplo;
\item \_  escreve em subscrito. Útil para escrever subíndice, por exemplo;
\item \{\} define configurações dos comandos \LaTeX\ ;
\end{itemize}
Devido às características interpretativas desses caracteres dentro do compilador \LaTeX\ , eles devem ser tratados com muito cuidado na hora do texto ser escrito. Caso necessário escrever esses caracteres especias, é necessário utilizar comandos de escape ou, como utilizado neste documento, usar a opção \textit{verbatim} para mostrar exemplos de código.

\section{Comandos \LaTeX\ }
\subsection{Cabeçalho: Preâmbulo}
O preâmbulo contém dados iniciais, como o título do documento, a inserção dos nomes dos autores e seus respectivos dados. Este "cabeçalho" também define o tamanho do papel a ser utilizado (caso for impresso), bem como as configurações de língua para o compilador \LaTeX\ não se confundir ao encontrar caracteres especiais, como os encontrados na Língua Portuguesa.

\subsubsection{Classe de Documento}
Sintaxe:
\begin{verbatim}
\documentclass{papel, fonte}
\end{verbatim}

O comando documentclass é requerido para todos os documentos gerados no \LaTeX\ . É ela quem define qual o tamanho de papel que vai ser utilizado e qual é o tamanho da fonte do documento em si.
O padrão para o argumento papel é \textit{a4paper} e a fonte é \textit{12pt}.

\subsubsection{Título}
Sintaxe:
\begin{verbatim}
\title{Título}
\end{verbatim}

O comando title também é requerido para o bom funcionamento do \LaTeX\ . Como já está propriamente apontado, este comando é quem define o título deste documento.

\subsubsection{Autores}
Sintaxe:
\begin{verbatim}
\author{
	linha 1\\
	linha 2\\
	...
	linha n\\
}
\end{verbatim}

O comando author é tambem necessário para os documentos \LaTeX\ . Ela define quem são os autores do documento.
Este comando suporta entrada de várias linhas: ao final de cada um, deve-se adicionar \textbackslash newline ou \textbackslash \textbackslash .

\subsubsection{Data}
Sintaxe:
\begin{verbatim}
\date{data}
\end{verbatim}

O comando data define a data de geração do documento \LaTeX\ . Pode ser escrito tanto em forma de números, ou como uma frase.

\hspace{0.1pt}Alternativamente, alguns autores utiliza o comando \textbackslash today dentro deste comando, retornando a data de geração do documento em extenso. Portanto, esta saída é dependente dos pacotes de regionalização utilizados no documento; caso omitido, é utilizado, por padrão, a língua francesa.

\subsection{Documento}

\subsubsection{Seções e Subseções}
Para criar uma nova seção, é necessário utilizar o comando section. Desta forma, o \LaTeX\ criará uma seção dita e, caso for requerido, será indexado automaticamente em um índice.

Sintaxe:
\begin{verbatim}
\section{Nome da Seção}
\end{verbatim}

Para inicializar um ambiente:
\begin{verbatim}
\begin{ambiente}
Seu texto
\end{ambiente}
\end{verbatim}

Para criar um documento:
\begin{verbatim}
\begin{document}
Seu documento
\end{document}
\end{verbatim}

Para criar uma lista:
\begin{verbatim}
\begin{tipo de lista}
\item item1;
\item item2;
.............
\item item n;
\end{tipo de lista}
\end{verbatim}

Para escrever em itálico: (\textbf{i}talic)
\begin{verbatim}
\textit{texto} 
\end{verbatim}
\[\textit {texto}\]

Para escrever em negrito: (\textbf{b}old\textbf{f}ace)
\begin{verbatim}
\textbf{texto} 
\end{verbatim}
\[\textbf {texto}\]

Para escrever letras gregas: 
\begin{verbatim}
$\alpha \beta \gamma$                   
$\Sigma \Delta \Omega$
\end{verbatim}
\[\alpha \beta \gamma\] 
\[\Sigma \Delta \Omega\]

Para escrever uma fórmula: 
\begin{verbatim}
$formula$
\end{verbatim}
\[f(x)= x^2\]

Para sobreescrever: 
\begin{verbatim}
$texto^{}{texto sobreescrito}$
\end{verbatim}
\[x^{2k+1}\]

Para subescrever: 
\begin{verbatim}
$texto_{texto sobreescrito}$ 
\end{verbatim}
\[x_1\]

Mutiplicação: 
\begin{verbatim}
$fator \times fator = resultado$ 
\end{verbatim}
\[3 \times 3 = 9\]

Divisão: 
\begin{verbatim}
\fator \div fator = resultado$ 
\end{verbatim} 
\[4 \div 2 = 2\]

Frações: 
\begin{verbatim}
$\frac{numerador}{denominador}$
\end{verbatim}
\[\frac{1}{2}\]

Raízes: 
\begin{verbatim}
$\sqrt[índice]{radicando}$
\end{verbatim}
\[\sqrt [n] {2}\]


\end{document}