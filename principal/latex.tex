% para comentar use esse símbolo de porcentagem

% Preâmbulo
\documentclass[a4paper,12pt]{article}

\usepackage[utf8]{inputenc}
\usepackage[T1]{fontenc}
\usepackage{geometry}

\usepackage[brazil]{babel}
%\usepackage[T1]{fontenc}
\usepackage{ae}

%\DeclareOption{brazil}{\input{portuges.ldf}}

\title{Como escrever em \LaTeX\ }
\author{
	Ernesto Yuiti Saito\\
	Hayato Fujii\\
	Helio Albano de Oliveira\\
	Luiz Henrique Jofre da Silva\\
	Marcos Okamura Rodrigues\\
}
\date{\today}

% Fim preâmbulo

% Início documento em si
\begin{document}

% Faça o título conforme definido no title-preâmbulo
\maketitle

% Espaçamento 1,5 entre linhas
%\renewcommand{\baselinestretch}{1.5}

\section{Observações}
\hspace{2cm}Os seguintes caracteres têm um significado especial no \LaTeX\ :
\begin{itemize}
\item \textbackslash \ especifica um comando;
\item \& separa colunas;
\item \$ especifica comandos matemáticos;
\item \% ignora linha (comentários);
\item \^{} escreve em sobreescrito. Útil para escrever potências, por exemplo;
\item \_  escreve em subscrito. Útil para escrever subíndice, por exemplo;
\item \{\} define configurações dos comandos \LaTeX\ ;
\end{itemize}
\hspace{2cm}Devido às características interpretativas desses caracteres dentro do compilador \LaTeX\ , eles devem ser tratados com muito cuidado na hora do texto ser escrito. Caso necessário escrever esses caracteres especias, é necessário utilizar comandos de escape ou, como utilizado neste documento, usar a opção \textit{verbatim} para mostrar exemplos de código.

\section{Comandos \LaTeX\ }
\subsection{Cabeçalho: Preâmbulo}
\hspace{2cm}O preâmbulo contém dados iniciais, como o título do documento, a inserção dos nomes dos autores e seus respectivos dados. Este "cabeçalho"  \ \ também define o tamanho do papel a ser utilizado (caso for impresso), bem como as configurações de língua para o compilador \LaTeX\ não se confundir ao encontrar caracteres especiais, como os encontrados na Língua Portuguesa.

\subsubsection{Classe de Documento}
Sintaxe:
\begin{verbatim}
\documentclass{papel, fonte}
\end{verbatim}

\hspace{2cm}O comando documentclass é requerido para todos os documentos gerados no \LaTeX. É ela quem define qual o tamanho de papel que vai ser utilizado e qual é o tamanho da fonte do documento em si. O padrão para o argumento papel é \textit{a4paper} e a fonte é \textit{12pt}.

\subsubsection{Título}
Sintaxe:
\begin{verbatim}
\title{Título}
\end{verbatim}

\hspace{2cm}O comando title também é requerido para o bom funcionamento do \LaTeX\ . Como já está propriamente apontado, este comando é quem define o título deste documento.

\subsubsection{Autores}
Sintaxe:
\begin{verbatim}
\author{
	linha 1\\
	linha 2\\
	...
	linha n\\
}
\end{verbatim}

\hspace{2cm}O comando author é tambem necessário para os documentos \LaTeX. Ela define quem são os autores do documento.
Este comando suporta entrada de várias linhas: ao final de cada um, deve-se adicionar \textbackslash newline ou \textbackslash \textbackslash .

\subsubsection{Data}
Sintaxe:
\begin{verbatim}
\date{data}
\end{verbatim}

\hspace{2cm}O comando data define a data de geração do documento \LaTeX\ . Pode ser escrito tanto em forma de números, ou como uma frase.

\hspace{2cm}Alternativamente, alguns autores utiliza o comando \textbackslash today dentro deste comando, retornando a data de geração do documento em extenso. Portanto, esta saída é dependente dos pacotes de regionalização utilizados no documento; caso omitido, é utilizado, por padrão, a língua francesa.

\subsection{Documento}

\subsubsection{Parágrafos e Subparágrafos}
\hspace{2cm}O comando \textbackslash paragraph \{texto\} e \textbackslash subparagraph \{texto\} geram, respecitvamente, um parágrafo e um subparágrafo. Nota-se que ambos não contém recuo inicial e estão destacados em negrito no texto.

\subsubsection{Seções e Subseções}
Sintaxe:
\begin{verbatim}
\section{Nome da Seção}
\subsection{Nome da Subseção}
\subsubsection{Nome da Subsubseção}
\end{verbatim}

\hspace{2cm}Para criar uma nova seção, é necessário utilizar o comando section. Desta forma, o \LaTeX\ criará uma seção dita e, caso for requerido, será indexado automaticamente em um índice (comando \textbackslash index). É possivel fazer isto em 3 níveis: seção, sub-seção e sub-sub-seção. Caso não seja necessário numeração da seção, utilize o comando \textbackslash (sub)(sub)section*.

\subsubsection{Capítulos}
Sintaxe:
\begin{verbatim}
\chapter{Nome do capítulo}
\end{verbatim}

\hspace{2cm}Semelhante à uma criação de uma nova seção, os capítulos também sao auto-enumerados e indexados corretamente pelo compilador \LaTeX\ .

\subsubsection{Criação}
Sintaxe:
\begin{verbatim}
\begin{document}
Seu documento
\end{document}
\end{verbatim}
\hspace{2cm}É o encapsulamento principal; todas os outros ambientes, exceto ao pre-âmbulo, devem ser inseridos dentro deste. Comandos inseridos fora deste ambiente não serão mostrados dentro do documentos. É de fundamental importância para qualquer documento \LaTeX\ .


\subsubsection{Ambiente}
Sintaxe:
\begin{verbatim}
\begin{Nome do Ambiente}
Seu texto
\end{Nome do Ambiente}
\end{verbatim}

\hspace{2cm}Um ambiente é basicamente um encapsulador do texto. Dentro dele é onde o texto deve ser escrito em si. Pode ser usado somente para "indexar", de forma interna, um certo pedaço de texto dentro do código \LaTeX\ .

\subsubsection{Listas}
Sintaxe:
\begin{verbatim}
\begin{Tipo de Lista}
\item item1;
\item item2;
...
\item item n;
\end{Tipo de Lista}
\end{verbatim}

\hspace{2cm}Cria listas em um ambiente. Existem 2 tipos de listas: \textit{enumerate} e \textit{itemize}. O último basta para listas simples, enquanto \textit{enumerate} cria listas enumeradas. Cada item deve ser colocado entre o ambiente após o comando \textbackslash. Estes podem ser detalhadas logo após utilizando o comando \textbackslash \textit{description <descrição>}.

\subsubsection{Formatação}

\textbf{Inserir espaços}: \textbackslash<espaço><espaço>...<espaço>\textbackslash \\
\textbf{Quebra de linha}: \textbackslash \textbackslash ou \textbackslash newline \\
\textbf{Inserir recuo à esquerda}: \textbackslash hspace\{espacamento na medida especificada\} \\
\textbf{Itálico (\textit{it}alic)}: \textbackslash textit\{texto\} \\
\textbf{Negrito} (\textbf{b}old\textbf{f}ace): \textbackslash textbf\{texto\} \\

\subsubsection{Fonte de letras}

%\textbf{Muito pequeno}: \textbackslash tiny - \tiny{pequeno}\\
\textbf{Tamanho de rodapé}: \textbackslash footnotesize - \footnotesize{rodapé}\\
\textbf{Normal}: \textbackslash normalsize - \normalsize{normal}\\
\textbf{Grande}: \textbackslash large - \large{grande}\\
%\textbf{Muito grande}: \textbackslash huge - \huge{muito grande}\\


\subsection{Alinhamento}

\subsubsection{Centralização}
Sintaxe:
\begin{verbatim}
\begin{center}
Texto à ser centralizado
\end{center}
\end{verbatim}

\begin{center}
Centraliza o texto desta forma.
\end{center}

\subsubsection{Alinhar à esquerda}
Sintaxe:
\begin{verbatim}
\begin{flushleft}
Texto a ser alinhado à esquerda
\end{flushleft}
\end{verbatim}

\begin{flushleft}
Alinha o texto à esquerda (padrão).
\end{flushleft}

\subsubsection{Alinhar à direita}
Sintaxe:
\begin{verbatim}
\begin{flushright}
Texto a ser alinhado à direita
\end{flushright}
\end{verbatim}

\begin{flushright}
Alinha o texto à direita, como demonstrado aqui.
\end{flushright}

\subsection{Comandos matemáticos}
\subsubsection{Letras gregas}

Sintaxe:
\begin{verbatim}
$\alpha \beta \gamma$                   
$\Sigma \Delta \Omega$
\end{verbatim}

\[\alpha \beta \gamma\] 
\[\Sigma \Delta \Omega\]

\hspace{2cm}Imprime letras gregas. Note que existem cifrões (\$) em ambos os lados do comando, pois o \LaTeX\ os trata como uma expressão matemática. Além disso, dependendo do \textit{case} do primeiro caracter da letra, pode-se usar diferentes capitalizações.

\subsubsection{Fórmulas matemáticas}
Sintaxe:
\begin{verbatim}
$formula$
\end{verbatim}

\[f(x)= x^2\]

\hspace{2cm}Para todas as fórmulas, é necessário que \$ esteja em ambos os lados da expressão em si. Pode ser necesário o uso de outros comandos complementares para escrever a expressão matemática completa.

Para sobreescrever
\begin{verbatim}
$texto^{}{texto sobreescrito}$
\end{verbatim}: \[x^{2k+1}\]

Para subescrever
\begin{verbatim}
$texto_{texto sobreescrito}$ 
\end{verbatim}: \[x_1\]

Mutiplicação
\begin{verbatim}
$fator \times fator = resultado$ 
\end{verbatim}: \[3 \times 3 = 9\]

Divisão
\begin{verbatim}
\fator \div fator = resultado$ 
\end{verbatim} 
\[4 \div 2 = 2\]

Frações
\begin{verbatim}
$\frac{numerador}{denominador}$
\end{verbatim}
\[\frac{1}{2}\]

Raízes
\begin{verbatim}
$\sqrt[índice]{radicando}$
\end{verbatim}
\[\sqrt [n] {2}\]

Binomiais
\begin{verbatim}
$n \choose {k \over 2}$
\end{verbatim}
\[n \choose {k \over 2} \]

Somatórios
\begin{verbatim}
$\sum_{i=1}^{N}x^2+3$
\end{verbatim}
\[ \sum_{i=1}^{N}x^2+3\]

Produtórios
\begin{verbatim}
$\prod_{y=0}^3 y - 1$
\end{verbatim}
\[ \prod_{y=0}^3 y - 1 \]

Integrais
\begin{verbatim}
$\[\int_{-\infty}^{+\infty} \frac{x}{2} + x^2 dx\]$
\end{verbatim}
\[\int_{-\infty}^{+\infty} \frac{x}{2} + x^2 dx\]

Limites
\begin{verbatim}
$\[\lim_{x \to \infty}{1 \over x}=0\]$
\end{verbatim}
\[\lim_{x \to \infty}{1 \over x}=0\]

\end{document}