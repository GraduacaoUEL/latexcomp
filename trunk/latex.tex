% para comentar use esse símbolo de porcentagem

\documentclass[a4paper,12pt]{article}

\usepackage[utf8]{inputenc}
\usepackage[T1]{fontenc}
\usepackage{geometry}

\usepackage[francais]{babel}

\title{Como escrever em \LaTeX\ ?}
\author{Ciência da Computação}
\date{}

\begin{document}
\maketitle

Como escrever em \LaTeX\ 

Para escrever em itálico utilize \textit {texto}

Para escrever em negrito utilize \textbf {texto}

Como escrever letras gregas

\[\alpha \beta \gamma \] 

\[ \Sigma \Delta \Omega \]

%	\[<fórmula> \]
Para escrever uma formula use 
\[f(x)= x^2\]


%	\[<texto>^{<texto sobreescrito>}\]
Para sobreescrever   
\[x^{2k+1}\]

%	\[<texto>_<texto subescrito>\]
Para subescrever 
\[x_1\]

%	\[\<fator> \times <fator> = <resultado>\]
Mutiplicação 
\[3 \times 3 = 9\]

%	\[\<dividendo> \div <divisor> = <resultado>\]
Divisão
 \[4 \div 2 = 2\]

%	\[\frac {<numerador>}{<denominador>} \]
Frações
\[\frac{1}{2}\]


%	\[\sqrt [índice] {radicando} \]
Raízes
\[\sqrt [n] {2} \]
\end{document}

